%!TeX root=main.tex

\section{Coordenadas esféricas}
Recordamos que el cambio de variable a coordenadas esféricas es
\begin{equation*}
 x=r\cos\alpha\sen\beta,
 \;
 y=r\sen\alpha\sen\beta,
 \;
 z=r\cos\beta,
 \quad
 r\geq0,
 \;
 \alpha\in[0,2\pi],
 \;
 \beta\in[0,\pi].
\end{equation*}
Por ejemplo para calcular el volumen de una esfera de
ecuación $x^2+y^2+z^2 = 9$
se debería calcular
\begin{equation*}
 \iint_{Q} f(r,\alpha,\beta)r^2\sen\beta\,\mathrm{d}A,
 \quad
 \text{ donde }
 Q = [0,3]\times[0,2\pi]\times[0,\pi],
 f(r,\alpha,\beta) = 1.
\end{equation*}
\begin{maximai}
 integrate(integrate(integrate(r^2*sin(%beta),r,0,3), %alpha, 0, 2*%pi),%beta,0,%pi);
\end{maximai}\begin{maximao}
 36\pi
\end{maximao}
