%!TeX root=main.tex

\section{Integrales dobles}
Para calcular integrales dobles se usa el comando
\maximain{integrate} dos veces, una por cada integración
respecto la variable correspondiente.

Por ejemplo para calcular
$\iint_{R} (xy+y+1)\,\mathrm{d}A$,
donde $R=[0,2]\times[0,1]$ se puede hacer lo siguiente.
\begin{maximai}
 integrate(integrate(x*y+y+1,x,0,2),y,0,1);
\end{maximai}\begin{maximao}
 4
\end{maximao}
También se pueden intercambiar el orden de integración.
\begin{maximai}
 integrate(integrate(x*y+y+1,y,0,1),x,0,2);
\end{maximai}\begin{maximao}
 4
\end{maximao}

Otro ejemplo, sería calcular la siguiente integral doble
\begin{equation*}
 \iint_{R} (-x^2y-x-3)\,\mathrm{d}A,
 \text{ donde }
 R = \{(x,y)\in\mathbb{R}^2 : 2\leq y\leq 2+x, 0\leq x\leq 1\}.
\end{equation*}
\begin{maximai}
 integrate(integrate(-x^2*y-x-3,y,2,2+x),x,0,1);
\end{maximai}\begin{maximao}
 -{{73}\over{30}}
\end{maximao}
