%!TeX root=main.tex

\section{Cálculo de longitud de una curva}
La longitud de la curva dada como la gráfica de una función
$f(x)$
en un intervalo
$[a,b]$
se calcula por la expresión
\begin{equation*}
 \int_{a}^{b} \sqrt{1+\left(f'(x)\right)^2}\,\mathrm{d}x.
\end{equation*}
Por ejemplo, para calcular la longitud del arco de curva de
$9y^2 = 4x^3$ considerando la parte con segunda coordenada positiva
para el intervalo $[0,3]$.
Se podría calcular como se muestra a continuación.
\begin{maximai}
 f(x):=sqrt(4/9*x^3)$
\end{maximai}\begin{maximai}
 integrate(sqrt(1+diff(f(x),x)^2),x,0,3);
\end{maximai}\begin{maximao}
 \frac{14}{3}
\end{maximao}
