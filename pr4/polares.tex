%!TeX root=main.tex

\section{Coordenadas polares}
Recordamos que el cambio de variable a coordenadas polares es
\begin{equation*}
 x=r\cos\alpha,
 \;
 y=r\sen\alpha,
 \quad
 r\geq0,
 \;
 \alpha\in[0,2\pi].
\end{equation*}
Por ejemplo para calcular el área de un círculo de radio $2$,
se debería calcular
\begin{equation*}
 \iint_{R} f(r,\alpha)r\,\mathrm{d}A,
 \quad
 \text{ donde }
 R = [0,2]\times[0,2\pi],
 f(r,\alpha) = 1.
\end{equation*}
\begin{maximai}
 integrate(integrate(r,r,0,2), %alpha, 0, 2*%pi);
\end{maximai}\begin{maximao}
 4\pi
\end{maximao}
