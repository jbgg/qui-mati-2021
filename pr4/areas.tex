%!TeX root=main.tex

\section{Cálculo de áreas}
El problema del cálculo de áreas se puede resolver con el uso
de integrales definidas.
Por ejemplo para calcular el área entre la parábola que se muestra
\begin{figure}
 \centering
 \begin{tikzpicture}[scale=1,domain=-2:2,samples=400]
  \draw[gray,domain=-2.1:2.1] plot (\x, 0);
  \draw[gray,domain=-.1:4.1] plot (0, \x);
  \foreach \i in {-2,...,2}{
   \draw (\i,-.1) -- (\i,.1) node[label=below:$\i$]{};
  }
  \foreach \i in {1,...,4}{
   \draw (-.1,\i) -- (.1,\i) node[right]{$\i$};
  }
  \draw[blue, domain=-2:2] plot (\x, {-(\x)^2+4});
 \end{tikzpicture}
 \caption{Gráfica de la parábola $y=-x^2+4$}
 \label{fig:gr1}
\end{figure}
en la gráfica de la Figura \ref{fig:gr1} con el eje de abscisa
entre los puntos $x=-2$ y $x=2$,
se puede hacer lo que se muestra en el siguiente comando.
\begin{maximai}
 integrate(-x^2+4,x,-2,2);
\end{maximai}\begin{maximao}
 \frac{32}{3}
\end{maximao}

Si se quiere calcular el área encerrada entre la gráfica de la
función $f(x) = \sen(x)$ y el eje de abscisa en el intervalo
$[0,2\pi]$ no se puede realizar simplemente con
\begin{maximai}
 integrate(sin(x),x,0,2*%pi);
\end{maximai}\begin{maximao}
 0
\end{maximao}
ya que el área sería cero,
pero como se puede ver en la Figura \ref{fig:gr2}
el área debe ser positiva.
En este caso hay que estudiar el signo de la función ya que
en algunos puntos la función es negativa.
Sabemos que la función se hace cero dentro del intervalo
$[0,2\pi]$ en los puntos
$x=0$, $x=\pi$ y $x=2\pi$.
Y la función es positiva en el intervalo $[0,\pi]$
y negativa en el intervalo $[\pi,2\pi]$
por tanto el área sería la siguiente.
\begin{maximai}
 integrate(sin(x),x,0,%pi)+integrate(-sin(x),x,%pi,2*%pi);
\end{maximai}\begin{maximao}
 4
\end{maximao}
\begin{figure}[hb]
 \centering
 \begin{tikzpicture}[scale=1,domain=0:2*pi,samples=400]
  \draw[gray,domain=0:2*pi] plot (\x, 0);
  \draw[gray,domain=-1:1] plot (0, \x);
  \draw (0,-.1) -- (0,.1) node[right,label=below:$0$]{};
  \draw (2*pi,-.1) -- (2*pi,.1) node[label=below:$2\pi$]{};
  \draw (-.1,-1) -- (.1,-1) node[right]{$-1$};
  \draw (-.1,1) -- (.1,1) node[right]{$1$};
  \draw[blue, domain=0:2*pi] plot (\x, {sin(\x r)});
 \end{tikzpicture}
 \caption{Gráfica de la función $y=\sen(x)$}
 \label{fig:gr2}
\end{figure}

Si se requiere calcular el área encerrada entre dos gráficas,
por ejemplo para las función $f(x)=x$ e $g(x)=x^2$,
se deberá buscar primero los puntos de corte de las gráficas.
Una manera de realizarlo sería resolver la ecuación
$x=x^2$.
\begin{maximai}
 f(x):=x$
\end{maximai}\begin{maximal}
 g(x):=x^2$
\end{maximal}\begin{maximai}
 solve(f(x)=g(x),x);
\end{maximai}\begin{maximao}
 \left[ x=0\operatorname{,}x=1\right]
\end{maximao}
Lo siguiente que habrá que saber es qué función está por arriba
y cuál está por abajo.
Para ello evaluamos en algún punto intermedio de cada intervalo
que se haya obtenido, en este caso solamente existe un intervalo.
\begin{maximai}
 is(f(.5)>g(.5));
\end{maximai}\begin{maximao}
 \mbox{%default
  true}
\end{maximao}
Por tanto el área se calculará con la siguiente expresión.
\begin{maximai}
 integrate(f(x)-g(x), x, 0, 1);
\end{maximai}\begin{maximao}
 \frac{1}{6}
\end{maximao}
