%!TeX root=main.tex

\section{Integrales triples}
El cálculo de integrales triples es análogo al de
integrales dobles. Veamos a continuación la resolución
de un par de ejemplos.

\begin{equation*}
 \iiint_{Q} (x+y+z+xyz)\,\mathrm{d}V,
 \text{ donde }
 Q = [0,1]\times[0,2]\times[0,1].
\end{equation*}
\begin{maximai}
 integrate(integrate(integrate(x+y+z+x*y*z,x,0,1),y,0,2),z,0,1);
\end{maximai}\begin{maximao}
 {{9}\over{2}}
\end{maximao}

\begin{equation*}
 \iiint_{Q} 6xyz\,\mathrm{d}V,
 \text{ donde }
 Q = \{(x,y,z)\in\mathbb{R}^3 : 0\leq x\leq 1, 0\leq y\leq 4-x,
 0\leq z\leq 4-x-y\}.
\end{equation*}
\begin{maximai}
 integrate(integrate(integrate(6*x*y*z,z,0,4-x-y),y,0,4-x),x,0,1);
\end{maximai}\begin{maximao}
 {{1909}\over{120}}
\end{maximao}
