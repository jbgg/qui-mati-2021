%!TeX root=main.tex

\section{Coordenadas cilíndricas}
Recordamos que el cambio de variable a coordenadas cilíndricas es
\begin{equation*}
 x=r\cos\alpha,
 \;
 y=r\sen\alpha,
 \;
 z=z,
 \quad
 r\geq0,
 \;
 \alpha\in[0,2\pi].
\end{equation*}
Por ejemplo para calcular el volumen de un cilindro de altura $3$
y de radio $\sqrt{5}$
se debería calcular
\begin{equation*}
 \iint_{Q} f(r,\alpha,z)r\,\mathrm{d}A,
 \quad
 \text{ donde }
 Q = [0,\sqrt{5}]\times[0,2\pi]\times[0,3],
 f(r,\alpha,z) = 1.
\end{equation*}
\begin{maximai}
 integrate(integrate(integrate(r,r,0,sqrt(5)), %alpha, 0, 2*%pi),z,0,3);
\end{maximai}\begin{maximao}
 15\pi
\end{maximao}
