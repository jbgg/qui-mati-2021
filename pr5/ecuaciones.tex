%!TeX root=main.tex

\section{Ecuaciones paramétricas e implícitas de un
subespacio}

Las ecuaciones paramétricas a partir de una base de
un subespacio vectorial se pueden escribir directamente
como la combinación lineal de la base usando los parámetros.

Por ejemplo para el subespacio vectorial de $\mathbb{R}^4$
con base $\{v_1,v_2\}$ donde
$v_1 = (1,2,1,1)$ y $v_2 = (0,1,0,-1)$, las ecuaciones
paramétricas usando los parámetros $a$ y $b$ se escribre en
maxima como mostramos a continuación.
\begin{maximai}
 v1:[1,2,1,1]$
\end{maximai}\begin{maximal}
 v2:[0,1,0,-1]$
\end{maximal}\begin{maximai}
 ecparam:[x,y,z,t]-a*v1-b*v2;
\end{maximai}\begin{maximao}
 \left[ x-a , y-b-2\,a , z-a , t+b-a \right]
\end{maximao}

Para pasar de ecuaciones paramétricas a ecuaciones
implícitas, se usará la técnica de eliminación de
paramétros, para ello se usa la orden \maximain{eliminate}.
\begin{maximai}
 ecimplic:eliminate(ecparam, [a,b]);
\end{maximai}\begin{maximao}
 \left[ y-3\,x+t , x-z \right]
\end{maximao}
De lo anterior deducimos que unas ecuaciones implícitas
del espacio vectorial anterior es
\begin{align*}
 y-3x+t & = 0\\
 x-z & = 0\\
\end{align*}

Para pasar de ecuaciones implícitas a ecuaciones paramétricas
basta usar el comando \maximain{linsolve}.
\begin{maximai}
 linsolve([y-3*x+t=0, x-z=0],[x,y,z,t]);
\end{maximai}\begin{maximao}
 \left[ x={\it \%r}_{2} , y=3\,{\it \%r}_{2}-{\it \%r}_{1} , z=
 {\it \%r}_{2} , t={\it \%r}_{1} \right]
\end{maximao}

Por último veamos veamos cómo deducir unas ecuaciones implícitas
directamente a partir de una base. La manera es escalonar la
siguiente matriz (no se debe usar \maximain{echelon} ya que depende
de parámetros).
\begin{maximai}
 A:transpose(matrix(v1,v2,[x,y,z,t]));
\end{maximai}\begin{maximao}
 \ifx\endpmatrix\undefined\pmatrix{\else\begin{pmatrix}\fi 1&0&x\cr
  2&1&y\cr 1&0&z\cr 1&-1&t\cr
 \ifx\endpmatrix\undefined}\else\end{pmatrix}\fi
\end{maximao}\begin{maximai}
 A1:rowop(A,2,1,A[2,1]);
\end{maximai}\begin{maximao}
 \ifx\endpmatrix\undefined\pmatrix{\else\begin{pmatrix}\fi 1&0&x\cr
  0&1&y-2\,x\cr 1&0&z\cr 1&-1&t\cr
 \ifx\endpmatrix\undefined}\else\end{pmatrix}\fi
\end{maximao}\begin{maximai}
 A2:rowop(A1,3,1,A[3,1]);
\end{maximai}\begin{maximao}
 \ifx\endpmatrix\undefined\pmatrix{\else\begin{pmatrix}\fi 1&0&x\cr
  0&1&y-2\,x\cr 0&0&z-x\cr 1&-1&t\cr
 \ifx\endpmatrix\undefined}\else\end{pmatrix}\fi
\end{maximao}\begin{maximai}
 A3:rowop(A2,4,1,A[4,1]);
\end{maximai}\begin{maximao}
 \ifx\endpmatrix\undefined\pmatrix{\else\begin{pmatrix}\fi 1&0&x\cr
  0&1&y-2\,x\cr 0&0&z-x\cr 0&-1&t-x\cr
 \ifx\endpmatrix\undefined}\else\end{pmatrix}\fi
\end{maximao}\begin{maximai}
 A4:rowop(A3,4,2,A[4,2]);
\end{maximai}\begin{maximao}
 \ifx\endpmatrix\undefined\pmatrix{\else\begin{pmatrix}\fi 1&0&x\cr
  0&1&y-2\,x\cr 0&0&z-x\cr 0&0&y-3\,x+t\cr
 \ifx\endpmatrix\undefined}\else\end{pmatrix}\fi
\end{maximao}
De las últimas dos filas de la tercera columna se deducen las ecuaciones
implícitas del subespacio vectorial.
