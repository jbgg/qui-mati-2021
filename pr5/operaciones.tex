%!TeX root=main.tex

\section{Operaciones y funciones de matrices}

\begin{itemize}
 \item \textbf{Suma y resta de matrices.} La suma de matrices se
  realiza el operador \maximain{+} y la resta de matrices con el
  operador \maximain{-}.
  \begin{maximai}
   A+B;
  \end{maximai}\begin{maximao}
   \ifx\endpmatrix\undefined\pmatrix{\else\begin{pmatrix}\fi 0&1&4\cr
    5&-1&-1\cr 1&3&-1\cr
   \ifx\endpmatrix\undefined}\else\end{pmatrix}\fi
  \end{maximao}
 \item \textbf{Multiplicación escalar.} El producto de un número por
  una matriz se realiza con \maximain{*}.
  \begin{maximai}
   2*A;
  \end{maximai}\begin{maximao}
   \ifx\endpmatrix\undefined\pmatrix{\else\begin{pmatrix}\fi 2&2&6\cr
   6&0&-2\cr 0&4&2\cr \ifx\endpmatrix\undefined}\else\end{pmatrix}\fi
  \end{maximao}
 \item \textbf{Multiplicación de matrices.} La multiplicación de dos
  matrices se lleva a cabo con el operador punto \maximain{.}.
  \begin{maximai}
   A.B;
  \end{maximai}\begin{maximao}
   \ifx\endpmatrix\undefined\pmatrix{\else\begin{pmatrix}\fi 4&2&-5
    \cr -4&-1&5\cr 5&-1&-2\cr
   \ifx\endpmatrix\undefined}\else\end{pmatrix}\fi
  \end{maximao}
 \item \textbf{Potencia de una matriz.} La potencia de una matriz
  se calcula con \maximain{\^}\maximain{\^}.
  \begin{maximai}
   A^^7;
  \end{maximai}\begin{maximao}
   \ifx\endpmatrix\undefined\pmatrix{\else\begin{pmatrix}\fi 2797&2077
    &2455\cr 1983&1664&1463\cr 1416&1322&1617\cr
   \ifx\endpmatrix\undefined}\else\end{pmatrix}\fi
  \end{maximao}
 \item \textbf{Matriz traspuesta.} La matriz traspuesta se calcula con
  \maximain{transpose}.
  \begin{maximai}
   transpose(A);
  \end{maximai}\begin{maximao}
   \ifx\endpmatrix\undefined\pmatrix{\else\begin{pmatrix}\fi 1&3&0\cr
   1&0&2\cr 3&-1&1\cr \ifx\endpmatrix\undefined}\else\end{pmatrix}\fi
  \end{maximao}
 \item \textbf{Matriz inversa.} La matriz inversa de una matriz se
  calcula con \maximain{invert}.
  \begin{maximai}
   invert(A);
  \end{maximai}\begin{maximao}
   \ifx\endpmatrix\undefined\pmatrix{\else\begin{pmatrix}\fi {{2
    }\over{17}}&{{5}\over{17}}&-{{1}\over{17}}\cr -{{3}\over{17}}&{{1
    }\over{17}}&{{10}\over{17}}\cr {{6}\over{17}}&-{{2}\over{17}}&-{{3
   }\over{17}}\cr \ifx\endpmatrix\undefined}\else\end{pmatrix}\fi
  \end{maximao}
 \item \textbf{Rango de una matriz.} El rango de una matriz se calcula
  con \maximain{rank}.
  \begin{maximai}
   rank(A);
  \end{maximai}\begin{maximao}
   3
  \end{maximao}
 \item \textbf{Determinante de una matriz.} El determinante de una
  matriz se calcula con \maximain{determinant}.
  \begin{maximai}
   determinant(A);
  \end{maximai}\begin{maximao}
   17
  \end{maximao}
 \item \textbf{Matriz identidad.} La matriz identidad de orden $n$
  se puede escribir como \maximain{ident(n)}.
  \begin{maximai}
   ident(4);
  \end{maximai}\begin{maximao}
   \ifx\endpmatrix\undefined\pmatrix{\else\begin{pmatrix}\fi 1&0&0&0
    \cr 0&1&0&0\cr 0&0&1&0\cr 0&0&0&1\cr
   \ifx\endpmatrix\undefined}\else\end{pmatrix}\fi
  \end{maximao}
\end{itemize}

También existe la manera de trabajar con elementos de
una matriz.
\begin{enumerate}
 \item \textbf{Elemento de la posición $(i,j)$.} Por ejemplo para
  obtener el elemento de la fila $2$ y columna $3$ de la matriz $A$
  se hace los siguiente.
  \begin{maximai}
   A[2,3];
  \end{maximai}\begin{maximao}
   -1
  \end{maximao}
 \item \textbf{Obtención de una fila.} Para obtener la segunda fila
  de la matriz $A$, se puede escribir.
  \begin{maximai}
   A[2];
  \end{maximai}\begin{maximao}
   \left[ 3 , 0 , -1 \right]
  \end{maximao}
 \item \textbf{Obtención de una columna.} Para obtener la tercera
  columna de la matriz $A$, se escribe.
  \begin{maximai}
   col(A, 3);
  \end{maximai}\begin{maximao}
   \ifx\endpmatrix\undefined\pmatrix{\else\begin{pmatrix}\fi 3\cr -1
   \cr 1\cr \ifx\endpmatrix\undefined}\else\end{pmatrix}\fi
  \end{maximao}
 \item \textbf{Obtención de una submatriz.} El comando para obtener
  una submatriz es
  \begin{center}
   \maximain{submatrix(i1,i2,...,A,j1,j2,...)}
  \end{center}
  que devuelve la submatriz de $A$ eliminando las filas
  \maximain{i1,i2,...} y las columnas \maximain{j1,j2,...}.
  \begin{maximai}
   submatrix(1,A,1,3);
  \end{maximai}\begin{maximao}
   \ifx\endpmatrix\undefined\pmatrix{\else\begin{pmatrix}\fi 0\cr 2
   \cr \ifx\endpmatrix\undefined}\else\end{pmatrix}\fi
  \end{maximao}
 \item \textbf{Agregado de filas.} El comando para agregar filas a
  una matriz $A$ es
  \begin{center}
   \maximain{addrow(A,f1,f2,...)}.
  \end{center}
  \begin{maximai}
   addrow(A,[1,3,1],[2,2,2]);
  \end{maximai}\begin{maximao}
   \ifx\endpmatrix\undefined\pmatrix{\else\begin{pmatrix}\fi 1&1&3\cr
    3&0&-1\cr 0&2&1\cr 1&3&1\cr 2&2&2\cr
   \ifx\endpmatrix\undefined}\else\end{pmatrix}\fi
  \end{maximao}
 \item \textbf{Agregado de columnas.} El comando para agregar columnas
  a una matriz $A$ es
  \begin{center}
   \maximain{addcol(A,f1,f2,...)}.
  \end{center}
  \begin{maximai}
   addcol(A,[1,3,1],[2,2,2]);
  \end{maximai}\begin{maximao}
   \ifx\endpmatrix\undefined\pmatrix{\else\begin{pmatrix}\fi 1&1&3&1&2
    \cr 3&0&-1&3&2\cr 0&2&1&1&2\cr
   \ifx\endpmatrix\undefined}\else\end{pmatrix}\fi
  \end{maximao}
\end{enumerate}
