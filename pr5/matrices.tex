%!TeX root=main.tex

\section{Matrices}

Para definir las matrices se usa el comando \maximain{matrix}
definiendo las filas, por ejemplo la matriz
\begin{equation*}
 A = \begin{pmatrix}
  1 & 1 & 3 \\
  3 & 0 & -1 \\
  0 & 2 & 1 \\
 \end{pmatrix}
\end{equation*}
se define como
\begin{maximai}
 A:matrix([1,1,3],[3,0,-1],[0,2,1]);
\end{maximai}\begin{maximao}
 \ifx\endpmatrix\undefined\pmatrix{\else\begin{pmatrix}\fi 1&1&3\cr
 3&0&-1\cr 0&2&1\cr \ifx\endpmatrix\undefined}\else\end{pmatrix}\fi
\end{maximao}
Y la matriz 
\begin{equation*}
 B = \begin{pmatrix}
  -1 & 0 & 1 \\
  2 & -1 & 0 \\
  1 & 1 & -2 \\
 \end{pmatrix}
\end{equation*}
se define como
\begin{maximai}
 B:matrix([-1,0,1],[2,-1,0],[1,1,-2]);
\end{maximai}\begin{maximao}
 \ifx\endpmatrix\undefined\pmatrix{\else\begin{pmatrix}\fi -1&0&1
  \cr 2&-1&0\cr 1&1&-2\cr
 \ifx\endpmatrix\undefined}\else\end{pmatrix}\fi
\end{maximao}
