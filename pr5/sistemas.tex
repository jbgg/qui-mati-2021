%!TeX root=main.tex

\section{Sistemas de ecuaciones lineales}

Para resolver sistemas de ecuaciones lineales se usa el
comando \maximain{linsolve}. Esta función admite dos argumentos,
el primer argumento es la lista de ecuaciones y el segundo es
la lista de las incógnitas.
\begin{maximai}
 linsolve([x-2*y-z = 1, x-y=1, 2*x+y-3*z=4],[x,y,z]);
\end{maximai}\begin{maximao}
 \left[ x={{4}\over{3}} , y={{1}\over{3}} , z=-{{1}\over{3}}
 \right]
\end{maximao}
En el ejemplo anterior el sistema es compatible determinado, ya
que se obtiene una única solución.
\begin{maximai}
 linsolve([x-2*y-z = 1, x-y=1, 2*x+y-3*z=4,x+y+z=1],[x,y,z]);
\end{maximai}\begin{maximao}
 \left[\  \right]
\end{maximao}
El sistema del ejemplo anterior es incompatible, ya que no existe
solución.
\begin{maximai}
 linsolve([x-2*y-z = 1, x-y=1, 2*x-3*y-z=2],[x,y,z]);
\end{maximai}\begin{maximao}
 \left[ x=1-{\it \%r}_{1} , y=-{\it \%r}_{1} , z={\it \%r}_{1}
 \right]
\end{maximao}
El sistema anterior es compatible indeterminado, ya que la solución
depende de un parámetro \maximain{\%r1}.
