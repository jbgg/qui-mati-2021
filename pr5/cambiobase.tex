%!TeX root=main.tex

\section{Cambio de base}

Las coordenadas de un vector respecto de una base
se pueden calcular usando la matriz de cambio de base.
Si un vector $v$ tiene coordenadas $(a_1,\ldots, a_n)$
en la base $B'=\{v_1,\ldots,v_n\}$, ¿cómo se calculan
las coordenadas del vector $v$ respecto otra base $B$?
Para ello habrá que calcular la matriz de cambio de base.

Veamos un ejemplo. Se tiene una base $B'=\{v_1,v_2,v_3\}$
de $\mathbb{R}^3$ siendo los vectores de la base (respecto
la base canónica)
\begin{align*}
 v_1 & = (1,0,1), \\
 v_2 & = (-1,1,0), \\
 v_3 & = (0,1,-1). \\
\end{align*}
Y el vector $v=(4,1,-2)$ está dado en las coordenadas respecto
la base canónica. ¿Cuáles son las coordenadas del vector
$v$ respecto la base $B'$?
Para ello formamos la matriz cambio de base de la base $B'$ a
la base canónica, escribiendo los vectores de la base $B'$
respecto la base canónica en columnas en la matriz $M$.
\begin{maximai}
 v1:[1,0,1]$
\end{maximai}\begin{maximal}
 v2:[-1,1,0]$
\end{maximal}\begin{maximal}
 v3:[0,1,-1]$
\end{maximal}\begin{maximai}
 M:transpose(matrix(v1,v2,v3));
\end{maximai}\begin{maximao}
 \ifx\endpmatrix\undefined\pmatrix{\else\begin{pmatrix}\fi 1&-1&0
  \cr 0&1&1\cr 1&0&-1\cr
 \ifx\endpmatrix\undefined}\else\end{pmatrix}\fi
\end{maximao}
Ya que la matriz $M$ convierte vectores en coordenadas en la base
$B'$ a la base canónica habrá que calcular $M^{-1}$ que será
la matriz de cambio de base de las coordenadas canónica a las
coordenadas en la base $B'$. De esta manera se puede calcular
el vector $v$ en las coordenadas de la base $B'$.
\begin{maximai}
 v:[4,1,-2]$
\end{maximai}\begin{maximai}
 invert(M).v;
\end{maximai}\begin{maximao}
 \ifx\endpmatrix\undefined\pmatrix{\else\begin{pmatrix}\fi {{3
  }\over{2}}\cr -{{5}\over{2}}\cr {{7}\over{2}}\cr
 \ifx\endpmatrix\undefined}\else\end{pmatrix}\fi
\end{maximao}
