%!TeX root=main.tex

\section{Dependencia e independencia lineal}

Una posibilidad para estudiar la dependencia lineal o
independencia lineal de un conjunto de vectores es
formar la matriz cuya fila son los vectores y estudiar
el rango, si el rango coincide con el número de vectores
entonces los vectores son linealmente independiente en
otro caso serán linealmente dependientes.

\begin{maximai}
 u1:[1,0,2,3]$
\end{maximai}\begin{maximal}
 u2:[2,1,4,6]$
\end{maximal}\begin{maximal}
 u3:[4,0,9,13]$
\end{maximal}\begin{maximal}
 u4:[3,-1,7,10]$
\end{maximal}\begin{maximai}
 A:matrix(u1,u2,u3,u4);
\end{maximai}\begin{maximao}
 \ifx\endpmatrix\undefined\pmatrix{\else\begin{pmatrix}\fi 1&0&2&3
  \cr 2&1&4&6\cr 4&0&9&13\cr 3&-1&7&10\cr
 \ifx\endpmatrix\undefined}\else\end{pmatrix}\fi
\end{maximao}\begin{maximai}
 rank(A);
\end{maximai}\begin{maximao}
 3
\end{maximao}

Otra manera de estudiar la dependencia lineal entre vectores sería
usar la orden \maximain{echelon}. La ventaja de este último comando
es que también nos da información sobre una base del sistema generado
por los vectores.
\begin{maximai}
 echelon(A);
\end{maximai}\begin{maximao}
 \ifx\endpmatrix\undefined\pmatrix{\else\begin{pmatrix}\fi 1&0&2&3
  \cr 0&1&0&0\cr 0&0&1&1\cr 0&0&0&0\cr
 \ifx\endpmatrix\undefined}\else\end{pmatrix}\fi
\end{maximao}
