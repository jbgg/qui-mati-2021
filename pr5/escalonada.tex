%!TeX root=main.tex

\section{Forma escalonada reducida por filas}

El comando \maximain{echelon} calcula una matriz escalonada
equivalente por filas a otra matriz.
\begin{maximai}
 A:matrix([1,3,4,-5],[2,-2,1,3],[2,2,1,-5])$
\end{maximai}\begin{maximai}
 Aesc:echelon(A);
\end{maximai}\begin{maximao}
 \ifx\endpmatrix\undefined\pmatrix{\else\begin{pmatrix}\fi 1&3&4&-5
  \cr 0&1&{{7}\over{8}}&-{{13}\over{8}}\cr 0&0&1&{{3}\over{7}}\cr
 \ifx\endpmatrix\undefined}\else\end{pmatrix}\fi
\end{maximao}
La matriz anterior es escalonada pero no es escalonada reducida, ya
que no hay ceros por encima de los pivotes. Para conseguir una matriz
escalonada reducida se tendrá que aplicar transformaciones elementales
por filas.
\begin{maximai}
 Aesc1:rowop(Aesc,1,3,Aesc[1,3]);
\end{maximai}\begin{maximao}
 \ifx\endpmatrix\undefined\pmatrix{\else\begin{pmatrix}\fi 1&3&0&-{{
   47}\over{7}}\cr 0&1&{{7}\over{8}}&-{{13}\over{8}}\cr 0&0&1&{{3
 }\over{7}}\cr \ifx\endpmatrix\undefined}\else\end{pmatrix}\fi
\end{maximao}
\begin{maximai}
 Aesc2:rowop(Aesc1,2,3,Aesc1[2,3]);
\end{maximai}\begin{maximao}
 \ifx\endpmatrix\undefined\pmatrix{\else\begin{pmatrix}\fi 1&3&0&-{{
   47}\over{7}}\cr 0&1&0&-2\cr 0&0&1&{{3}\over{7}}\cr
 \ifx\endpmatrix\undefined}\else\end{pmatrix}\fi
\end{maximao}
\begin{maximai}
 Aesc3:rowop(Aesc2,1,2,Aesc2[1,2]);
\end{maximai}\begin{maximao}
 \ifx\endpmatrix\undefined\pmatrix{\else\begin{pmatrix}\fi 1&0&0&-{{
   5}\over{7}}\cr 0&1&0&-2\cr 0&0&1&{{3}\over{7}}\cr
 \ifx\endpmatrix\undefined}\else\end{pmatrix}\fi
\end{maximao}
